\chapter{Bluetooth Low Energy}
\section{Characteristics}
\begin{longtable}{c|p{2.1in}|c}
\textbf{Friendly name} & \multicolumn{1}{|c|}{\textbf{Description}} & \textbf{UUID} \\ \hline

Turn Heat Off&
Write a value of 0 to turn the heating element off&
10110010-5354-4f52-5a26-4249434b454c \\ \hline

Turn Heat On&
Write a value of 0 to turn the heating element on&
1011000f-5354-4f52-5a26-4249434b454c \\ \hline

Turn Fan Off &
Write a value of 0 to turn the fan off &
10110014-5354-4f52-5a26-4249434b454c\\ \hline

Turn fan On&
Write a value of 0 to turn the fan on &
10110013-5354-4f52-5a26-4249434b454c\\ \hline

Heat/Fan Register&
Stores the \textbf{state for heat} and \textbf{fan}.
\textbf{Subscribe} to this characteristic to receive events
when the state of the heating element or the fan changes&
1010000c-5354-4f52-5a26-4249434b454c \\ \hline 

Settings Register&
Configure the Volcano's \textbf{Display While Cooling} and
set the Volcano's display temperature to \textbf{\f\ or \c.}
\textbf{Subscribe} to this characteristic to receive events
when the device changes between \f\ and \c\ &
 1010000d-5354-4f52-5a26-4249434b454c \\ \hline

More Settings&
Configure the Volcano to vibrate (pulse the fan) when it reaches it's target temperature. 
Write the mask $0x400$ converted to a 32bit byte array to turn this setting on. 
Write the mask $0x10000 + 0x400$ to turn this setting off. 
A bitwise `and' resulting in a answer equal to 0 means this setting is on&
1010000e-5354-4f52-5a26-4249434b454c \\ \hline

BLE Firmware&
Read and decode the value to ``utf-8'' to get the Volcano's current Bluetooth firmware version&
10100004-5354-4f52-5a26-4249434b454c\\ \hline

FirmwareVersion&
Read and decode the value to ``utf-8'' to get the Volcano's current firmware version&
10100003-5354-4f52-5a26-4249434b454c\\ \hline

Serial Number&
Read, decode the value to ``utf-8'', and 
substring the first 8 characters to get the Volcano's serial number&
10100008-5354-4f52-5a26-4249434b454c\\ \hline

Hours Of Operation&
Read and convert the value to a UInt16.
This is usually used with minutes of operation to get the full usage time&
10110015-5354-4f52-5a26-4249434b454c\\ \hline

Minutes Of Operation&
Read and convert the value to a UInt16.
This is usually used with hours of operation to get the full usage time &
10110016-5354-4f52-5a26-4249434b454c\\ \hline

Current Temperature&
\texttt{\Huge To Do}&
10110001-5354-4f52-5a26-4249434b454c\\ \hline

Set Temperature&
Sets the temperature that the Volcano will heat up to when the heat is on.
When the value is read convert it to a UInt16, divide by 10, and round to the nearest int. 
To set the temperature multiply the desired temperature in \c\ by 10
and convert it to a 32bit byte array and write that to the characteristic.
This characteristic emits events when the temperature is changed on the Volcano.
You can change the ones place after multiplying by ten to write with increased precision.
You cannot read with increased precision at the time of writing this document&
10110003-5354-4f52-5a26-4249434b454c\\ \hline

Set off timer&
This setting is the starting value for ``Auto Off Time'' when the heating element is turned on.
The value can be read the same was as ``Auto Off Time'' and this value can be updated by
writing a 16bit byte array with the new value in seconds.  
The min value of this is 0 and the max value is assumed to be the max value of the data type.
Note this functionality is limited to 15-360 minutes to mirror the officially supported functionality.
Yes setting it to 0 is the biggest troll and results in the Volcano immediately turning off after you
turn it on. 
&
1011000d-5354-4f52-5a26-4249434b454c\\ \hline

Auto Off Time&Lets the caller know how long until the Volcano automatically turns itself off.
The value is stored on the Volcano in seconds.
To get the value in minutes read from the characteristic,
convert the value to \textbf{UInt16}, and divide by 60.
This characteristic \textbf{does not} emit events. 
Updated values must be polled or calculated.&
1011000c-5354-4f52-5a26-4249434b454c\\ \hline

Screen Brightness&Stores and Sets the display brightness.
Accepted values are 0-100.  
Set the display to 0 to turn it off. 
When the value is read convert it to a UInt16. 
When writing the brightness convert the value to a 16 bit byte array&
10110005-5354-4f52-5a26-4249434b454c\\ \hline
\end{longtable}
\section{Known Issues}
\begin{itemize}
    \item The characteristics for reading temperatures and receiving temperature updates all round to the nearest \c
        \subitem This means we cannot fully support all temperatures in \f\ because we cannot reliably read
        the values.  However we are able to write with full precision. 
    \item A successful write update to the Volcano's screen brightness sometimes results in the Volcano not updating its brightness
    \item The Volcano sometimes doesn't free up its Bluetooth connection. This can be fixed by power cycling the Volcano's Bluetooth by holding down the ``-'' button and the ``AIR'' button at the same time.
\end{itemize}

